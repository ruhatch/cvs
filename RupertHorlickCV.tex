\documentclass[DIV=15,color=DodgerBlue4]{komacv}

% \usepackage{CormorantGaramond}
\usepackage{quattrocento}
\usepackage{opensans}
\usepackage[light]{Chivo}
\usepackage[T1]{fontenc}

% \addtokomafont{firstnamefont}{\fontfamily{fos}\selectfont\scshape}

\newcommand\HUge{\fontsize{19}{60}\selectfont}

\renewcommand*\firstname{R{\HUge UPERT}\,}
\renewcommand*\familyname{H{\HUge ORLICK}}
\renewcommand*\mobile{+1 (778) 980-4817}
\renewcommand*\email{ruperthorlick@gmail.com}
\renewcommand*\addressstreet{4-2050 Vine Street}
\renewcommand*\addresscity{Vancouver, BC, V6K 3K1}
\renewcommand*\homepage{www.github.com/ruhatch}

\renewcommand\mobilesymbol{\,}
\renewcommand\emailsymbol{\,}

\setlength\dbitemmaincolwidth{12.4em}
\setlength\aftertitlevspace{-2em}
\setlength\hintscolwidth{0.15\textwidth}

\pagestyle{scrheadings}
\clearscrheadfoot

\begin{document}

  \rmfamily

  \maketitle

  \justify
  I am a Computer Scientist and Software Engineer with experience in functional
  programming and its mathematical foundations. I would like to bring the
  Computer Science perspective on distributed systems of information to new
  fields including, but not limited to, biology, politics, and sociology.

  \raggedright

  \vspace{-0.5em}

  \section{Experience}

    \cventry[1em]{Sep 2017 -- Present}{Lead Software
      Developer}{Money\&Co}{}{}{The company had accrued large amount of
      technical debt and brought me on to rebuild from scratch. Built new
      infrastructure using Nix for reproducible builds. Used Haskell for the
      server, the frontend, and the type-safe API for communicating between
      the two. Used GHCJS to compile the frontend targeting the web with
      Haskell. Had complete control over development, design
      decisions, and my time.}

    \cventry[1em]{Mar -- Sep 2017}{Software Development Engineer}{Myrtle
      Software}{}{}{Started part-time during my Masters degree and moved to
      full-time in June. Worked on compiling neural networks to FPGAs, using
      modern tools such as Haskell and Nix. Joined as one of three developers
      and became a leading member as the team grew. Was offered a senior
      position, including responsibility for the team's productivity.}
  
    \cventry[1em]{Summer 2016}{Research Intern}{Microsoft Research
      Cambridge}{}{}{Worked under Simon Peyton-Jones on a project to add
      functions and rich data structures to Excel. Prototyped an interesting
      new data structure and prepared demos. The demos were presented to project
      managers in Redmond and were well received.}

  \vspace{-1.5em}

  \section{Education}

    \cventry[1em]{2016 -- 2017}{MEng Computer Science}{University of Cambridge}{}{Distinction}{
      Thesis --- Formalised the theory of Generalised Species in Homotopy Type
      Theory (HoTT) using Agda. Worked closely with leading researcher in the
      field, who offered me a research position to continue working with him. \\
      Modules --- Category Theory, Multicore Semantics \& Programming, Advanced
      Functional Programming, Distributed Games \& Strategies, Interactive
      Formal Verification.}
    \cventry[1em]{2013 -- 2016}{BA (Hons) Computer Science}{University of Cambridge}{}{1 (81\%, Rank: 8/81)}{Dissertation --- Built the Path ORAM cryptographic primitive in OCaml on MirageOS, to allow search over encrypted documents. I analysed the performance and security properties in the 10,000 word dissertation.}
    \cventry[1em]{2007 -- 2012}{High School}{St. Paul's School}{}{}{
    \begin{tabular}{@{}ll}
      A-Level & Computing --- A$^*$, Maths --- A$^*$, Further Maths --- A$^*$, Physics --- A$^*$ \\
      AS-Level & Chemistry --- A \\
      GCSE & 10 A$^*$s, 1 A
    \end{tabular}
    }

  \vspace{-1.5em}

  \section{Other}

    \cvdoubleitem{Technical Skills}{\small
      \begin{compactitem}
        \item Haskell, OCaml, Nix, Java
        \item git, Octave, \LaTeX
      \end{compactitem}}{Qualifications}{\small\vspace{-1.2em}
      \begin{compactitem}
        \item CSIA Level 1 Ski Instructor
        \item Grade 5 Drum Kit
        \end{compactitem}}

\end{document}
